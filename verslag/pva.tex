\documentclass{verslag}
\begin{document}
\titelpagina
\inhoudsopgave


\section{Inleiding} % (fold)
\label{sec:inleiding}

    \subsection{Doel van het PVA} % (fold)
    \label{sub:doel_van_het_pva}
    Dit Plan van Aanpak(PVA) beschrijft het project ``eat IT'' zal worden uitgevoerd. Het bevat de projectdefinitie, de projectaanpak, de projectorganisatie, de activiteiten die moeten worden uitgevoerd en verder alle standaarden en procedures die gelden binnen het project.
    % subsection doel_van_het_pva (end)

    \subsection{Doelgroep} % (fold)
    \label{sub:doelgroep}
    Het PVA is bestemd voor:
    \begin{itemize}
        \item Inge en Tim, eigenaars eat IT en opdrachtgevers
        \item Marijn Pool, ontwikkelaar
        \item René Kooi, ontwikkelaar
        \item Ronald van Dijk, begeleidend docent
        \item Henkjan Hekman, begeleidend docent
    \end{itemize}
    % subsection doelgroep (end)

    \subsection{Toepassingsgebied van het PVA} % (fold)
    \label{sub:toepassingsgebied_van_het_pva}
    Alles wat binnen de scope van ``eat It'' valt.
    % subsection toepassingsgebied_van_het_pva (end)

    \subsection{subsection name} % (fold)
    \label{sub:subsection_name}
    Het plan van aanpak beslaat de onderstaande onderwerpen:
    \begin{itemize}
        \item Inleiding
        \item Projectdefinitie
        \item Projectaanpak
        \item Projectorganisatie
    \end{itemize}
    % subsection subsection_name (end)
% section inleiding (end)


\section{Projectdefinitie} % (fold)
\label{sec:projectdefinitie}
    
    \subsection{Projectopdracht} % (fold)
    \label{sub:projectopdracht}
    Vanuit de opdrachtgever is er de wens om de processen en datastroom van het bedrijf eat IT in kaart te brengen en dit om te zetten naar een web applicatie. Één van de wensen van de opdrachtgever is om bestellingen via de website te laten lopen alsmede het voorraadbeheer. 
    % subsection projectopdracht (end)

    \subsection{Probleemstelling} % (fold)
    \label{sub:probleemstelling}
    De opdrachtgever heeft in verband met de onvrede over verliezen en onduidelijkheid ons ingeschakeld om de processen binnen het bedrijf beter te stroomlijnen. Onderdelen die misgaan zijn het voorraadbeheer, wat is er op voorraad en wat moet er besteld worden om het aan te vullen tot een goede hoeveelheid. Tevens verlopen de bestellingen nu nog met papieren bestelformulieren en telefonische bestellingen, dit kan efficiënter en het is dan ook een wens om te zorgen dat er via een gestroomlijnde website besteld kan gaan worden zodat de informatie direct in de keuken terechtkomt en hier ook het voorraadbeheer nauwkeuriger door wordt.
    % subsection probleemstelling (end)

    \subsection{Doelstelling} % (fold)
    \label{sub:doelstelling}
    Een gestroomlijnde webapplicatie met bestelfunctionaliteit en een goed voorraad en bestelbeheer.
    % subsection doelstelling (end)

    \subsection{Definitie projecteinde} % (fold)
    \label{sub:definitie_projecteinde}
    Het project is afgerond als de applicatie voldoet aan de volgende eisen:
    \begin{itemize}
        \item Gestroomlijnde en schaalbare website
        \item Voorraadbeheer met afname door orderverwerking
        \item Bestelmogelijkheid door het systeem direct te koppelen aan de leveranciers.
    \end{itemize}
    Een ander mogelijk projecteinde is een deadline aangegeven door de begeleidende docenten.
    % subsection definitie_projecteinde (end)

    \subsection{Uitgangspunten en randvoorwaarden} % (fold)
    \label{sub:uitgangspunten_en_randvoorwaarden}
    Dit project gaat uit van het feit dat de opdrachtgever een webserver heeft met de volgende eisen:
        \begin{itemize}
            \item (Hardwarematige eisen)
            \item Apache of NGIX webserver
            \item MySQL database
            \item PHP 5.4 of later
        \end{itemize}
    Verder zal dit project gebruik maken van het MineTurtle-framework, een PHP framework ontwikkeld door dezelfde ontwikkelaars van dit project.
    % subsection uitgangspunten_en_randvoorwaarden (end)

% section projectdefinitie (end)

\section{Projectaanpak} % (fold)
\label{sec:projectaanpak}
    \subsection{Benaderingswijze} % (fold)
    \label{sub:benaderingswijze}
    Het project zal bestaan uit 5 fasen:
    \begin{itemize}
        \item Inventariseren, kijken wat de eisen en wensen zijn van de klant en deze vastleggen.
        \item Ontwerpen, het ontwikkelen van een (digitale)huisttijl en het maken van een datamodel.
        \item Programmeren, het uitwerken van het datamodel en tevens het ontwikkelen van de website.
        \item Opleveren met praktijktest, de uitrol binnen het bedrijf om te kijken of het voldoet aan de gestelde eisen.
        \item Onderhoud, zorgen dat het systeem mee blijft gaan met de tijd en zorgen dat het meegroeit met de opdrachtgever.
    \end{itemize}
    % subsection benaderingswijze (end)
    \subsection{Op te leveren producten} % (fold)
    \label{sub:op_te_leveren_producten}
    \begin{itemize}
        \item Inventariseren: een eisenlijst
        \item Ontwerpen: de digitale huisstijl en een data(flow)model
        \item Programmeren: Huisstijl toegepast op de webapplicatie.
        \item Opleveren met praktijktest: Werkende digitale omgeving die voldoet aan eisen.
        \item Onderhoud: een constant verbeterend product
    \end{itemize}
    % subsection op_te_leveren_producten (end)
% section projectaanpak (end)
\end{document}
